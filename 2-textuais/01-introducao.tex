\chapter{Introdução}

A produção de textos acadêmicos exige um alto nível de organização, clareza e padronização. Trabalhos como artigos e teses frequentemente seguem normas rígidas de formatação, como as da ABNT, que este documento seguirá. Ferramentas tradicionais de edição de texto podem ser úteis para a escrita básica, mas tornam-se limitadas quando o documento cresce em complexidade. É nesse contexto que o \LaTeX\ se destaca como uma alternativa poderosa e flexível para a escrita acadêmica.

Diferente de editores convencionais baseados em \textit{WYSIWYG}\footnote{\textit{"What You See Is What You Get"}, são \textit{softwares} que permitem a edição de texto diretamente na interface do documento final, sendo exemplos as ferramentas Microsoft Word e LibreOffice Writer.}, o \LaTeX\ utiliza um sistema de marcação no qual o autor define a estrutura do documento usando comandos específicos. Isso permite separar a forma do conteúdo, garantindo um alto grau de controle sobre a formatação sem precisar ajustar manualmente cada elemento. Além disso, o \LaTeX\ é amplamente utilizado em comunidades científicas devido à sua capacidade de lidar eficientemente com equações matemáticas, tabelas, referências bibliográficas, figuras e formatação automática de capítulos, seções e subseções.

Este documento foi elaborado para introduzir um estudante leigo ao universo do \LaTeX\, apresentando suas principais funcionalidades e demonstrando como utilizá-las dentro de um modelo de tese baseado na classe \abnTeX. Essa classe é especialmente útil para documentos longos, pois oferece uma estruturação avançada, maior personalização e recursos adicionais que facilitam a produção de textos acadêmicos extensos.

Nos capítulos seguintes, serão abordados os elementos fundamentais do \LaTeX\ para escrita acadêmica, começando com a estrutura básica de um documento, passando pela inclusão de equações, tabelas, figuras, citações e trechos de código-fonte. Além disso, serão demonstradas boas práticas para organizar um trabalho de forma eficiente e profissional, aproveitando ao máximo os recursos da ferramenta.

O objetivo deste guia é proporcionar um primeiro contato amigável com o \LaTeX\ e ajudar estudantes e pesquisadores a utilizá-lo com confiança, permitindo a criação de documentos bem estruturados e visualmente agradáveis sem a necessidade de esforço excessivo com formatação manual. Ao final deste documento, espera-se que o leitor compreenda os conceitos essenciais do \LaTeX\ e consiga aplicá-los em seus próprios trabalhos acadêmicos.