\chapter{Elementos básicos}

\section{Tabelas}\label{sec:tabelas}
A tabela \ref{tab:exemplo} é um exemplo de tabela feita no \LaTeX. É indicado utilizar produzir tabelas em formato consistente com o \LaTeX, quando possível, em vez de colar como imagem, de modo a manter a coerência textuais.

Note que foi criado um diretório próprio para armazenar as tabelas do documento, para facilitar o gerenciamento das tabelas no documento e deixar nosso ambiente de trabalho mais limpo. Em vez de deixar o comando completo no meio do texto, você pode armazenar sua tabela em um arquivo \texttt{.tex} separado e incluí-lo no documento a partir do comando \texttt{\textbackslash input}. O mesmo serve para quaisquer outros pedaços de texto ou elementos que você deseje armazenar separadamente do seu ambiente principal de edição.

\begin{table}[hbt!]
    \centering
    \caption{Título da tabela}
    \begin{tabular}{ccc}
        \hline
        \textbf{Item} & \textbf{Quantidade} & \textbf{Preço (R\$)} \\  
        \hline
        Livro        & 2       & 50,00  \\  
        Caderno      & 5       & 10,00  \\  
        Caneta       & 3       & 3,50   \\  
        \hline
    \end{tabular}
    \caption*{Fonte: Elaboração própria}
    \label{tab:exemplo}
\end{table}

Informações sobre o uso de tabelas no \LaTeX\ podem ser encontradas em \href{https://www.overleaf.com/learn/latex/Tables}{Tables (Overleaf)}.

O Overleaf possui uma ferramenta de edição gráfica de tabelas no modo de edição visual. Outra ferramenta útil pode ser o \href{https://www.tablesgenerator.com/}{Tables Generator}.

Se a tabela tiver sido gerada a partir de algum software específico, confira se já existe alguma ferramenta para exportar sua tabela automaticamente para formato \LaTeX. A título de exemplo, ver opções para a linguagem \href{https://stackoverflow.com/questions/5465314/tools-for-making-latex-tables-in-r}{R (Stack Overflow)}.

\section{Expressões formais}
A fórmula \ref{eqn:newton} é um exemplo de equação numerada.

\begin{equation}
(x + a)^n = \sum_{k=0}^n \binom{n}{k} x^k a^{n - k}
\label{eqn:newton}
\end{equation}

A seguir, a demonstração do teorema de Bayes (equação \ref{eqn:bayes}).

\begin{align}
    P(A|B) &= \frac{P(A \cap B)}{P(B)}\\
    &= \frac{P(A \cap B)}{P(B)} \cdot \frac{P(A)}{P(A)}\\
    &= \frac{P(A \cap B)}{P(A)} \cdot \frac{P(A)}{P(B)}\\
    &= \frac{P(B|A) \cdot P(A)}{P(B)}\label{eqn:bayes}
\end{align}

Mais informações relevantes em \href{https://www.overleaf.com/learn/latex/Mathematical_expressions}{Mathematical expressions (Overleaf)}.

Para construir as fórmulas em formato compatível com \LaTeX, pode ser útil a ferramenta \href{https://editor.codecogs.com/}{CodeCogs}.

Para descobrir os comandos para os símbolos que precisar utilizar, basta desenhar o símbolo em \href{https://detexify.kirelabs.org/classify.html}{Detexify}, que te retornará o comando e pacote requerido (importar pacote no preâmbulo do \texttt{main.tex} caso não esteja importado).

\section{Citações}

\subsection{Citação direta curta}

A citação no fluxo do texto é feita com o comando \texttt{\textbackslash textcite} e a citação isolada é feita com o comando \texttt{\textbackslash cite}.

Exemplo no texto (com \texttt{\textbackslash textcite}):

Conforme \textcite[8]{churchill2012marketing}, “embora a orientação para a produção seja muito criticada por vários profissionais de marketing, há situações em que ela é apropriada.”

Exemplo no texto (com \texttt{\textbackslash cite}):

“Embora a orientação para a produção seja muito criticada por vários profissionais de marketing, há situações em que ela é apropriada.” \cite[8]{churchill2012marketing}

\section{Imagens}
A figura \ref{fig:telles} apresenta um exemplo simples de como importar uma imagem.

\begin{figure}[hbt!]
    \centering
    \includegraphics[width=.5\textwidth]{figuras/telles.pdf}
    \caption{Biblioteca Telles}
    \label{fig:telles}
\end{figure}

O \LaTeX\ possui especificações próprias para escolher onde a imagem se localizará, mas é possível escolher preferências particulares para a formatação.

No nosso exemplo, usamos a opção \texttt{[hbt!]}, que define:

\begin{description}
    \item[h] \textit{(here)} a primeira prioridade é colocar a imagem na ordem em que aparece no \texttt{.tex}.
    \item[b] \textit{(bottom)} a segunda prioridade é colocar a imagem na parte inferior da página.
    \item[t] \textit{(top)} a terceira prioridade é colocar a imagem na parte superior da página.
    \item[!] \textit{(override)} força a substituição das preferências do \LaTeX\ pelas definidas.
\end{description}

Com o pacote \texttt{float}, é possível utilizar o parâmetro \texttt[H], para forçar a inserção da imagem exatamente na ordem em que aparece no \texttt{.tex}, dado que mesmo \texttt{[h!]} pode não surtir o mesmo efeito\footnote{Ainda assim, dê preferência a testar \texttt{[h!]} antes.}, caso seja entendido pelo programa que isso pode gerar problemas. Se ainda assim o usuário quiser utilizar esse novo parâmetro, esteja ciente de eventuais erros com a construção do documento, incluindo um tempo necessário de compilação maior.

Mais informações, incluindo a implementação modos de visualização mais complexos, como múltiplas imagens, podem ser encontradas em \href{https://www.overleaf.com/learn/latex/Inserting_Images}{Inserting Images (Overleaf)}.
