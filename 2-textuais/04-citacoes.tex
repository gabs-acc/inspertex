\chapter{Citações}
O gerenciamento de citações e referências é um aspecto particularmente crítico de um trabalho acadêmico. Para bem abordar esse tema, dedicamos um capítulo próprio às citações no \LaTeX, com exemplos que devem ser suficientes para a maioria das citações que o usuário vá precisar utilizar.

\section{Exemplos do Manual Insper}

\subsection{Exemplos preliminares}

A citação no texto pode é feito na forma \textcite{rego1994pureza}.

Caso haja duas citações com mesmos nome de autor e ano de publicação, as citações são diferenciadas automaticamente com letras\footnote{Essa diferenciação também ocorre nas referências ao final do documento.}, como em \textcite{rego1994usina}.

Conforme \textcite[8]{churchill2012marketing}, “embora a orientação para a produção seja muito criticada por vários profissionais de marketing, há situações em que ela é apropriada.”

"Embora a orientação para a produção seja muito criticada por vários
profissionais de marketing, há situações em que ela é apropriada.” \parencite[8]{churchill2012marketing} 

\subsection{Citação direta longa}

\textcite[8]{churchill2012marketing} explicam que:

\begin{citacao}
Embora a orientação para a produção seja muito criticada
por vários profissionais de marketing, há situações em
que ela é apropriada. Por exemplo, em mercados de alta
tecnologia com mudanças rápidas, muitas vezes não há
tempo suficiente realizar pesquisas de marketing a fim de
perguntar aos clientes o que eles querem.
\end{citacao}

ou

\begin{citacao}
Embora a orientação para a produção seja muito criticada
por vários profissionais de marketing, há situações em
que ela é apropriada. Por exemplo, em mercados de alta
tecnologia com mudanças rápidas, muitas vezes não há
tempo suficiente realizar pesquisas de marketing a fim de
perguntar aos clientes o que eles querem \parencite{churchill2012marketing}.
\end{citacao}

\subsection{Citação indireta}

Na visão de \textcite{osterwalder2011business} deve existir preocupação
com custos em quaisquer modelos de negócios, contudo as estruturas de baixo
custo são mais importantes para alguns tipos de negócios. Podemos dividir as
estruturas em duas: negócios direcionados pelos custos e negócios direcionados
pelo valor.

ou

A preocupação com custos é inerente a quaisquer modelos de negócios, contudo
as estruturas de baixo custo são mais importantes para alguns tipos de negócios.
Podemos dividir as estruturas em duas: negócios direcionados pelos custos e negócios
direcionados pelo valor \parencite{osterwalder2011business}.

\subsection{Citação de citação}

Para \textapud{simon1997administrative}[161]{mintzberg2010safari}  “o segredo da
resolução de problemas é que não existe segredos. Ela é realizada por meio de
complexas estruturas de elementos simples e conhecidos”.

\subsection{Citações de informações não publicadas}

No texto:

A empresa mantinha sete centros de revenda, prestando serviços exclusivos para duas
fábricas de produção de plástico biodegradável instaladas nas cidades de Diadema, São
Bernardo e Santo André\footfullcite{biolife2008}.

ou

Conforme relatório de descrição de atividade comercial da  \citeauthor{biolife2008}
publicado em \citedate{biolife2008}, a empresa mantinha sete centros de revenda,
prestando serviços exclusivos para duas fábricas de produção de plástico biodegradável
instaladas nas cidades de Diadema, São Bernardo e Santo André.

\subsection{Citação de fontes não paginadas}

De acordo com \textcite[cap. 1]{cnseg2017mercado}, durante os 800 anos que se passaram,
as duas atividades (seguro e bancária) permaneceram trocando conhecimentos entre si e
desenvolveram muito e até os nossos dias se aprimoraram mutuamente.

ou

Durante os 800 anos que se passaram, as duas atividades (seguro e bancária) permaneceram
trocando conhecimentos entre si e desenvolveram muito e até os nossos dias se aprimoraram
mutuamente \parencite[cap. 1]{cnseg2017mercado}\footnote{Fonte não paginada}.

\section{Mais alguns exemplos}

Referir-se ao arquivo \texttt{.tex} para ver como as citações são criadas.

\begin{itemize}
  \item \cite[ver:][30--42]{churchill2012marketing}
  \item \textcites[142-146]{churchill2012marketing}[21--23]{biolife2008}[45]{simon1997administrative}
  \item \cites[142-146]{churchill2012marketing}[21--23]{biolife2008}[45]{simon1997administrative}
  \item \cite{nouri2025}
  \item \cite{marginalrevolutionBordaCount}
  \item \apud{simon1997administrative}[161]{mintzberg2010safari}
\end{itemize}