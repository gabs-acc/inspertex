\chapter{Testando opções de glossário e índice}

Referência na forma \gls{latex}.

Referência na forma \gls{html}.

Abaixo, exemplo do \textit{template} canônico do pacote \texttt{glossaries}.

The \texttt{glossaries} package automatically generates a list of glossary entries. It's great for keeping track of your \gls{domain-knowledge} and \glspl{tla}. In this example we've put the glossary definitions in a separate \texttt{glossary.tex} file, which you can edit via the project menu.

In this example, several keywords\index{keywords} will be used 
which are important and deserve to appear in the Index\index{Index}.

Terms like generate\index{generate} and some\index{others} will also 
show up. Terms in the index can also be nested \index{Index!nested}
