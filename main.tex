% ====================================================================================
% ESTE É O DOCUMENTO PRINCIPAL DO PROJETO
%
% - É este o arquivo que o LaTeX buscará compilar primeiro para gerar o documento final.
% - O conteúdo aqui organizado definirá a estrutura geral do seu trabalho.
% - As opções do documento (nome de autores, informações sobre o texto, etc.),
%   serão definidas aqui
% - É aqui que você escolherá quais elementos textuais incluir ou omitir da versão
%   final do documento
%
% IMPORTANTE: não remover linhas com comandos essenciais, como 'begin{document}' e
%             'end{document}' e evitar alterar as definições do documento anteriores
%             às informações omplementares, a não ser que saiba bem o que está fazendo
%             (permitido incluir pacotes conforme necessidade, mas checar se há
%             incompatibilidade com as pré-definições do documento)
%
% Obs.: este modelo utiliza o pacote 'fontspec' para carregar as fontes 'Arial' e
%       'Times New Roman'. Para que ele funcione, é necessário compilar o documento
%       com XeLaTeX (recomendado) ou LuaLaTeX. No Overleaf, é possível trocar o
%       compilador na aba 'Menu'.
% ====================================================================================

\documentclass[
    12pt, % tamanho da fonte
    openright, % numeração começa apenas na primeira página textual
    oneside, % impressão apenas em anverso (para imprimir em anverso e reverso, opção: twoside)
    a4paper, % formato da folha
    %draft,                 % descomentar para compilação mais veloz (figuras e links internos não carregam)
    chapter=TITLE, % títulos em maiúsculas
    section=Title, % seções sem formatação adicional
    subsection=Title, % subseções sem formatação adicional
    subsubsection=Title, % (sub-)subseções sem formatação adicional
    spanish, % suporte a idioma para hifenização
    english, % suporte a idioma para hifenização
    brazil % suporte a idioma para hifenização (último idioma declarado é o principal do documento)
]{insper-abntex2}

% ----
% Importando pacotes
% ----

\usepackage{xcolor} % controle de cores
\usepackage{graphicx} % inclusão de gráficos e imagens
\usepackage{microtype} % melhorias de tipografia, espaçamento e justiificação
\usepackage{caption} % legendas
\usepackage{imakeidx} % criação de índices
\makeindex % compilar índice
\usepackage[
    acronym, % configurações específicas para acrônimos
    xindy={language=portuguese}, % para o índice remissivo
    subentrycounter, % contagem de subentradas no glossário
    seeautonumberlist, % ver também números automáticos na lista
    nonumberlist=true % sem numeração das entradas no glossário
]{glossaries} % gerenciamento de glossários
\newglossaryentry{latex}{%
  name=LaTeX,%
  description={Um sistema de preparação de documentos}}

\newglossaryentry{html}{%
  name=HTML,%
  description={Linguagem de marcação para páginas web}}

\newglossaryentry{domain-knowledge}{%
  name={domain knowledge},%
  description={valid knowledge used to refer to an area of human endeavour, an autonomous computer activity, or other specialized discipline}} % definições de glossário
\makeglossaries % compilar glossário

\usepackage{
    fontspec
} % uso de fontes TrueType (requer motor de compilação XeTeX ou LuaTeX)

% Para bibliografia
\usepackage{csquotes} % gerenciamento de citações
\usepackage[
    backend=biber, % backend de processamento do arquivo .bib
    style=abnt, % estilo de citações e referências
    sorting=nyt, % ordena as referências em nome, ano e título, respectivamente
    extrayear, % diferencia os anos com letras na bibliografia
    repeatfields, % imprime os campos repetidos na bibliografia
]{biblatex}
\renewcommand*{\mkbibnamegiven}[1]{#1} % prenome dado não é modificado
\renewcommand*{\mkbibnamefamily}[1]{#1} % sobrenome dado não é modificado
\DeclareFieldFormat{url}{Dispon\'ivel em: \url{#1}} % exibição da URL nas referências

% Pacotes para uso matemático (deletar e incluir pacotes aqui conforme necessidade)
\usepackage{amsmath} % equações avançadas
\usepackage{amssymb} % símbolos adicionais
\usepackage{amsfonts} % fontes adicionais
\usepackage{amsthm} % teoremas (criação e personalização)
\usepackage{mathtools} % extensão de amsmath

\usepackage{
    lipsum
} % geração de texto dummy (apenas para teste de formatação; permitido excluir)

% ----
% Configurações de aparência do PDF final
% ----

% informações do PDF (não editar)
\makeatletter
\hypersetup{%
    %pagebackref=true,
    pdftitle={\@title},
    pdfauthor={\@author},
    pdfsubject={\imprimirpreambulo},
    pdfcreator={LaTeX with abnTeX2},
    pdfkeywords={abnt, latex, abntex, abntex2, trabalho acadêmico, Insper},
    colorlinks=true,
    linkcolor=black, % cor dos links internos
    citecolor=black, % cor dos links das citações
    filecolor=magenta, % cor dos links dos arquivos
    urlcolor=gray, % cor dos links das URL's
    bookmarksdepth=4
}
\makeatother

% Configurações de espaçamento (não editar)
% Indentação dos parágrafos
\setlength{\parindent}{0pt}
% Espaçamento entre linhas
\setlength{\baselineskip}{12pt}
% Espaçamento entre parágrafos
\setlength{\parskip}{18pt}
% Espaçamento entre referências
\setlength{\bibitemsep}{18pt}

% ----
% Informações Complementares
% ----

% Edite conforme necessidade

% Fonte (descomentar a fonte a ser utilizada)
\setmainfont{Arial}
%\setmainfont{Times New Roman}

% Fontes bibliográficas
\addbibresource{referencias.bib}

% Título
\titulo{Título do trabalho}
% Subtítulo
\subtitulo{Subtítulo se houver}
% Autor (nome completo)
\autor{Seu Nome Completo}
% Ano com 4 dígitos
\data{\the\year} % \the\year imprime o ano corrente
% Nome da instituição
\instituicao{Insper}
% Nome do curso
\curso{Nome Completo do Curso}
% Localidade da apresentação do documento
\local{São Paulo}
% Orientador (nome completo)
\orientador[Orientador:]{Nome do Seu Orientador}
% Coorientador (nome completo)
\coorientador[Coorientador:]{Nome do Seu Coorientador}
% Tipo de trabalho (TCC  / Dissertação / Tese)
\tipotrabalho{Tipo de trabalho}
% Palavras-chaves (importante: separar por vírgulas)
\palavraschaves{abnt, LaTeX, formatação, equações, microeconomia}
% Palavras-chaves em inglês (importante: separar por vírgulas)
\keywords{abnt, LaTeX, formatting, equations, microeconomics}
% Professores convidados (importante: separar por vírgulas)
\professores{Adam Smith, John Maynard Keynes, Robert Lucas Jr.}

% Preâmbulo
\preambulo{(TCC / Dissertação / Tese) apresentado ao Curso (Graduação em Economia / Graduação em Administração / Graduação em Direito / Graduação em Engenharia (Computação/Mecânica/Mecatrônica) / Graduação em Ciência de Dados / Mestrado em Administração / LLM / Doutorado) como requisito parcial para a obtenção do título de (Bacharel / Mestre / Especialista / Pós-graduado / Doutor) em área de estudo.}

% ----
% Início do documento
% ----
\begin{document}
    % Retira espaço extra obsoleto entre as frases
    \frenchspacing

    % Elementos pré-textuais
    \pretextual
    % Capa (obrigatório)
    \imprimircapa
    % Folha de rosto (obrigatório)
    \imprimirfolhaderosto

    % Ficha catalográfica (opcional)
    \imprimirfichacatalografica
    % Errata (opcional)
    \begin{errata}
Sua errata aqui.
\end{errata}
    % Folha de aprovação (obrigatório)
    \imprimirfolhadeaprovacao
    % Dedicatória (opcional)
    \begin{dedicatoria}
\vspace*{\fill}
\centering
\noindent
\textit{
% Editar aqui
Ao verme que primeiro roeu\\
as frias carnes do meu cadáver\\
dedico como saudosa lembrança\\
estas Memórias Póstumas.
% ----
}
\vspace*{\fill}
\end{dedicatoria}}
    % Agradecimentos (opcional)
    \begin{agradecimentos}
Dedico este trabalho a X, Y, Z.
\end{agradecimentos}
    % Epígrafe (opcional)
    \begin{epigrafe}
\vspace*{\fill}
\begin{flushright}
\textit{
% Editar aqui
``Se Deus tivesse desejado que\\
houvesse mais de dois fatores de produção,\\
Ele teria tornado mais fácil a\\
criação de diagramas tridimensionais.\\
Robert M. Solow
% ----
}
\end{flushright}
\end{epigrafe}

    % Resumo (obrigatório)
    \begin{resumo}
Seu resumo aqui.

\textbf{Palavras-chaves:} \imprimirpalavraschavesresumo.
\end{resumo}
    % Abstract - Resumo em língua estrangeira (obrigatório)
    \begin{resumo}[Abstract]
 \begin{otherlanguage*}{english}
   Your abstract here.

    \textbf{Keywords:} \thekeywordsabstract.
 \end{otherlanguage*}
\end{resumo}
    % Lista de ilustrações (opcional)
    \imprimirlistadeilustracoes
    % Lista de tabelas (opcional)
    \imprimirlistadetabelas
    % Lista de abreviaturas e siglas (opcional)
    \begin{siglas} \itemsep -1pt
  \item[ABNT] Associação Brasileira de Normas Técnicas
  \item[abnTeX] ABsurdas Normas para \TeX
  \item[GNU] GNU's Not Unix
\end{siglas}
    % Lista de símbolos (opcional)
    \begin{simbolos} \itemsep -1pt
  \item[$ \Gamma $] Letra grega Gama
  \item[$ \Lambda $] Lambda
  \item[$ \zeta $] Letra grega minúscula zeta
  \item[$ \in $] Pertence
\end{simbolos}
    % Sumário (obrigatório)
    \imprimirsumario

    % Elementos textuais
    \textual
    % Capítulos do texto (incluir conforme necessidade)
    \chapter{Introdução}

A produção de textos acadêmicos exige um alto nível de organização, clareza e padronização. Trabalhos como artigos e teses frequentemente seguem normas rígidas de formatação, como as da ABNT, que este documento seguirá. Ferramentas tradicionais de edição de texto podem ser úteis para a escrita básica, mas tornam-se limitadas quando o documento cresce em complexidade. É nesse contexto que o \LaTeX\ se destaca como uma alternativa poderosa e flexível para a escrita acadêmica.

Diferente de editores convencionais baseados em \textit{WYSIWYG}\footnote{\textit{"What You See Is What You Get"}, são \textit{softwares} que permitem a edição de texto diretamente na interface do documento final, sendo exemplos as ferramentas Microsoft Word e LibreOffice Writer.}, o \LaTeX\ utiliza um sistema de marcação no qual o autor define a estrutura do documento usando comandos específicos. Isso permite separar a forma do conteúdo, garantindo um alto grau de controle sobre a formatação sem precisar ajustar manualmente cada elemento. Além disso, o \LaTeX\ é amplamente utilizado em comunidades científicas devido à sua capacidade de lidar eficientemente com equações matemáticas, tabelas, referências bibliográficas, figuras e formatação automática de capítulos, seções e subseções.

Este documento foi elaborado para introduzir um estudante leigo ao universo do \LaTeX\, apresentando suas principais funcionalidades e demonstrando como utilizá-las dentro de um modelo de tese baseado na classe \abnTeX. Essa classe é especialmente útil para documentos longos, pois oferece uma estruturação avançada, maior personalização e recursos adicionais que facilitam a produção de textos acadêmicos extensos.

Nos capítulos seguintes, serão abordados os elementos fundamentais do \LaTeX\ para escrita acadêmica, começando com a estrutura básica de um documento, passando pela inclusão de equações, tabelas, figuras, citações e trechos de código-fonte. Além disso, serão demonstradas boas práticas para organizar um trabalho de forma eficiente e profissional, aproveitando ao máximo os recursos da ferramenta.

O objetivo deste guia é proporcionar um primeiro contato amigável com o \LaTeX\ e ajudar estudantes e pesquisadores a utilizá-lo com confiança, permitindo a criação de documentos bem estruturados e visualmente agradáveis sem a necessidade de esforço excessivo com formatação manual. Ao final deste documento, espera-se que o leitor compreenda os conceitos essenciais do \LaTeX\ e consiga aplicá-los em seus próprios trabalhos acadêmicos.
    \chapter{Estrutura básica de um documento}

A estrutura de um documento em \LaTeX\ é composta por diferentes partes que organizam o conteúdo e garantem que a formatação seja aplicada corretamente. Neste capítulo, exploramos os principais elementos dessa estrutura, explicando sua função e como utilizá-los de forma eficiente.

É mister garantir que o arquivo-alvo para formatação seja o correto. No nosso \textit{template} de exemplo, é o arquivo denominado como \texttt{main.tex}.

\section{O preâmbulo do documento}
O preâmbulo do seu documento será a parte destinada e definir a estrutura deste. Aqui não é descrito o texto como aparecerá na versão final da sua tese, mas serão definidas as normas de formatação, bem como parâmetros que podem ser utilizados no restante no texto.

\subsection{Definição da classe}
O parâmetro de classe no \LaTeX\ define as configurações gerais de formatação do seu documento. Note que estamos importando as configurações definidas no arquivo \texttt{insper-abntex2.cls}. A modificação ou exclusão desse arquivo te impossibilitará de trabalhar adequadamente com as normas definidas para o nosso \textit{layout}.

A classe é definida na forma \texttt{\textbackslash documentclass[opções]\{classe\}}. A classe utilizada para nosso \textit{template} é o já definido \texttt{insper-abntex2}, e as opções modificam o comportamento da nossa classe. Além das opções já definidas explicitamente no \textit{template}, a maioria das opções para as classes \href{https://linorg.usp.br/CTAN/macros/latex/contrib/memoir/memman.pdf}{memoir} e \href{https://br.mirrors.cicku.me/ctan/macros/latex/contrib/abntex2/doc/abntex2.pdf}{abntex2} devem funcionar.

\subsection{Importando Pacotes}
Os pacotes adicionam recursos ao \LaTeX. Além dos pacotes importados como dependências da própria classe de documento definida no início do preâmbulo, é possível adicionar outros a depender da necessidade.

No preâmbulo do presente documento, já foram importados pacotes de uso frequente. Eles poderão ser excluídos do preâmbulo caso não precisem ser utilizados. Sua forma é \texttt{\textbackslash usepackage[opções]\{pacote\}}. Pesquise a documentação dos pacotes caso queira modificar seu comportamento.

\subsection{Informações sobre o seu documento}
Após as definições de formatação e funcionalidades usadas no seu documento com \LaTeX, você deverá também escrever as informações principais do seu documento. Todas essas informações são utilizadas em elementos obrigatórios do documento, portanto se certifique de preencher todos os campos corretamente. Caso algum campo esteja vazio, o documento ainda poderá ser formatado sem erros, mas deixando de imprimir esses campos. Cabe ao usuário prestar a devida atenção ao preenchimento devido.

\subsubsection{Observação sobre o nome do autor}
Na ficha catalográfica (caso a impressão desta esteja habilitada), é impresso o nome do autor com destaque ao último nome. Por exemplo, se for definido \texttt{\textbackslash autor\{Thomas John Sargent\}}, será impressa a forma "Sargent, Thomas John".

Caso mais de um nome deva ser destacado, indica-se separar esses últimos sobrenomes por "\texttt{\textbackslash\ }". A definição de \texttt{\textbackslash autor\{Robert Emerson Lucas Jr.\}} gera a forma "Jr., Robert Emerson Lucas". Em vez disso, defina \texttt{\textbackslash autor\{Robert Emerson Lucas\textbackslash\ Jr.\}}, para gerar a forma "Lucas Jr., Robert Emerson".

Como alternativas a \texttt{\textbackslash autor}, também estão disponíveis as macros \texttt{\textbackslash autora} e \texttt{\textbackslash author}, com a mesma funcionalidade.

\section{Ambiente \textit{document}}
Definidas as opções iniciais, é hora de organizar a apresentação do documento em si. Essa parte é feita entre \texttt{\textbackslash begin\{document\}} e \texttt{\textbackslash end\{document\}}, que definem respecticamente o início e o fim do ambiente do documento.

Para melhor organização do texto, dado que estamos tratando de documentos longos, é indicado que cada elemento seja definido em arquivo separado e chamado para seu arquivo principal.

\subsection{Elementos pré-textuais}
Os elementos importados com alguma macro que apresente prefixo \texttt{\textbackslash imprimir}\footnote{\texttt{\textbackslash imprimircapa}, \texttt{\textbackslash imprimirfolhaderosto}, etc.} dependem apenas das informações definidas no preâmbulo. Sua escolha será apenas a de escolher manter ou excluir/comentar esses elementos no seu texto, caso sejam opcionais.

Os elementos que dependem de algum outro texto específico serão importados na forma \texttt{\textbackslash input\{arquivo\}}. Este comando apenas insere o conteúdo do arquivo selecionado no exato ponto onde é chamado.\footnote{Motivo pelo qual é essencial manter ao menos uma linha em branco entre \texttt{\textbackslash imprimirfolhaderosto} e \texttt{\textbackslash imprimirfichacatalografica}, para evitar erros de formatação.}

O leitor atento já deve imaginar que o comando \texttt{\textbackslash input} também pode ser útil para organizar os textos dos capítulos do seu documento, de modo a tornar os arquivos \texttt{.tex} correspondentes menos inchados e facilitar a edição de elementos específicos destes. Isso poderá ser especialmente útil para inclusão de tabelas\footnote{Exemplo na seção \ref{sec:tabelas}.}, longas expressões matemáticas e outros elementos complexos.

\subsection{Elementos textuais}
Tão simples quanto poderia, aqui só será necessário importar para seu arquivo principal os arquivos referentes a cada um dos capítulos utilizados, em ordem de aparição no documento.

Aqui, no entanto, é indicada a inserção desses elementos na forma \texttt{\textbackslash include\{arquivo\}}, que não apenas insere o conteúdo do arquivo, mas também realiza formatações como elementos próprios. Por isso mesmo, ao contrário de \texttt{\textbackslash input}, a macro \texttt{\textbackslash include} pode ser chamada apenas no ambiente \textit{document}.

\section{Elementos pós-textuais}
Aqui, o usuário já deve conhecer suficientemente o modo de organização de um \texttt{.tex} típico.

O único adendo a ser feito aqui é o de que, na utilização de apêndices e anexos, é necessário manter manter os ambientes definidos para cada, nas formas:

\begin{verbatim}
\begin{apendicesenv}
    \partapendices
    \input{nome-do-apendice-a}
    \input{nome-do-apendice-b}
    \input{nome-do-apendice-c}
\end{apendicesenv}
\end{verbatim}

e

\begin{verbatim}
\begin{anexosenv}
    \partanexos
    \input{nome-do-anexo-a}
    \input{nome-do-anexo-b}
    \input{nome-do-anexo-c}
\end{anexosenv}
\end{verbatim}

Deve ser alterada apenas a parte em que os arquivos são inseridos, com \texttt{\textbackslash input}.

Para não imprimir esses elementos, algumas das opções do usuário são:

\begin{itemize}
    \item excluir os ambientes por completo (podendo escrever novamente mais tarde);
    \item apenas comentar todas as linhas da forma usual, com \%;
    \item deixar os ambientes entre \texttt{\textbackslash iffalse} e \texttt{\textbackslash fi}, fazendo o programa ignorar tudo o que estiver no meio.
\end{itemize}
    \chapter{Elementos básicos}

\section{Tabelas}\label{sec:tabelas}
A tabela \ref{tab:exemplo} é um exemplo de tabela feita no \LaTeX. É indicado utilizar produzir tabelas em formato consistente com o \LaTeX, quando possível, em vez de colar como imagem, de modo a manter a coerência textuais.

Note que foi criado um diretório próprio para armazenar as tabelas do documento, para facilitar o gerenciamento das tabelas no documento e deixar nosso ambiente de trabalho mais limpo. Em vez de deixar o comando completo no meio do texto, você pode armazenar sua tabela em um arquivo \texttt{.tex} separado e incluí-lo no documento a partir do comando \texttt{\textbackslash input}. O mesmo serve para quaisquer outros pedaços de texto ou elementos que você deseje armazenar separadamente do seu ambiente principal de edição.

\begin{table}[hbt!]
    \centering
    \caption{Título da tabela}
    \begin{tabular}{ccc}
        \hline
        \textbf{Item} & \textbf{Quantidade} & \textbf{Preço (R\$)} \\  
        \hline
        Livro        & 2       & 50,00  \\  
        Caderno      & 5       & 10,00  \\  
        Caneta       & 3       & 3,50   \\  
        \hline
    \end{tabular}
    \caption*{Fonte: Elaboração própria}
    \label{tab:exemplo}
\end{table}

Informações sobre o uso de tabelas no \LaTeX\ podem ser encontradas em \href{https://www.overleaf.com/learn/latex/Tables}{Tables (Overleaf)}.

O Overleaf possui uma ferramenta de edição gráfica de tabelas no modo de edição visual. Outra ferramenta útil pode ser o \href{https://www.tablesgenerator.com/}{Tables Generator}.

Se a tabela tiver sido gerada a partir de algum software específico, confira se já existe alguma ferramenta para exportar sua tabela automaticamente para formato \LaTeX. A título de exemplo, ver opções para a linguagem \href{https://stackoverflow.com/questions/5465314/tools-for-making-latex-tables-in-r}{R (Stack Overflow)}.

\section{Expressões formais}
A fórmula \ref{eqn:newton} é um exemplo de equação numerada.

\begin{equation}
(x + a)^n = \sum_{k=0}^n \binom{n}{k} x^k a^{n - k}
\label{eqn:newton}
\end{equation}

A seguir, a demonstração do teorema de Bayes (equação \ref{eqn:bayes}).

\begin{align}
    P(A|B) &= \frac{P(A \cap B)}{P(B)}\\
    &= \frac{P(A \cap B)}{P(B)} \cdot \frac{P(A)}{P(A)}\\
    &= \frac{P(A \cap B)}{P(A)} \cdot \frac{P(A)}{P(B)}\\
    &= \frac{P(B|A) \cdot P(A)}{P(B)}\label{eqn:bayes}
\end{align}

Mais informações relevantes em \href{https://www.overleaf.com/learn/latex/Mathematical_expressions}{Mathematical expressions (Overleaf)}.

Para construir as fórmulas em formato compatível com \LaTeX, pode ser útil a ferramenta \href{https://editor.codecogs.com/}{CodeCogs}.

Para descobrir os comandos para os símbolos que precisar utilizar, basta desenhar o símbolo em \href{https://detexify.kirelabs.org/classify.html}{Detexify}, que te retornará o comando e pacote requerido (importar pacote no preâmbulo do \texttt{main.tex} caso não esteja importado).

\section{Citações}

\subsection{Citação direta curta}

A citação no fluxo do texto é feita com o comando \texttt{\textbackslash textcite} e a citação isolada é feita com o comando \texttt{\textbackslash cite}.

Exemplo no texto (com \texttt{\textbackslash textcite}):

Conforme \textcite[8]{churchill2012marketing}, “embora a orientação para a produção seja muito criticada por vários profissionais de marketing, há situações em que ela é apropriada.”

Exemplo no texto (com \texttt{\textbackslash cite}):

“Embora a orientação para a produção seja muito criticada por vários profissionais de marketing, há situações em que ela é apropriada.” \cite[8]{churchill2012marketing}

\section{Imagens}
A figura \ref{fig:telles} apresenta um exemplo simples de como importar uma imagem.

\begin{figure}[hbt!]
    \centering
    \includegraphics[width=.5\textwidth]{figuras/telles.pdf}
    \caption{Biblioteca Telles}
    \label{fig:telles}
\end{figure}

O \LaTeX\ possui especificações próprias para escolher onde a imagem se localizará, mas é possível escolher preferências particulares para a formatação.

No nosso exemplo, usamos a opção \texttt{[hbt!]}, que define:

\begin{description}
    \item[h] \textit{(here)} a primeira prioridade é colocar a imagem na ordem em que aparece no \texttt{.tex}.
    \item[b] \textit{(bottom)} a segunda prioridade é colocar a imagem na parte inferior da página.
    \item[t] \textit{(top)} a terceira prioridade é colocar a imagem na parte superior da página.
    \item[!] \textit{(override)} força a substituição das preferências do \LaTeX\ pelas definidas.
\end{description}

Com o pacote \texttt{float}, é possível utilizar o parâmetro \texttt[H], para forçar a inserção da imagem exatamente na ordem em que aparece no \texttt{.tex}, dado que mesmo \texttt{[h!]} pode não surtir o mesmo efeito\footnote{Ainda assim, dê preferência a testar \texttt{[h!]} antes.}, caso seja entendido pelo programa que isso pode gerar problemas. Se ainda assim queira utilizar esse novo parâmetro, esteja ciente de eventuais erros com a construção do documento, incluindo um tempo necessário de compilação maior.

Mais informações, incluindo a implementação modos de visualização mais complexos, como múltiplas imagens, podem ser encontradas em \href{https://www.overleaf.com/learn/latex/Inserting_Images}{Inserting Images (Overleaf)}.
    \usepackage{listings}
\chapter{Testando Modos de Citação}
\section{Exemplos do Manual Insper}

\subsection{Exemplos preliminares}

A citação no texto pode é feito na forma \textcite{rego1994pureza}.

Caso haja duas citações com mesmos nome de autor e ano de publicação, as citações são diferenciadas automaticamente com letras\footnote{Essa diferenciação também ocorre nas referências ao final do documento.}, como em \textcite{rego1994usina}.

Conforme \textcite[8]{churchill2012marketing}, “embora a orientação para a produção seja muito criticada por vários profissionais de marketing, há situações em que ela é apropriada.”

"Embora a orientação para a produção seja muito criticada por vários
profissionais de marketing, há situações em que ela é apropriada.” \parencite[8]{churchill2012marketing} 

\subsection{Citação direta longa}

\textcite[8]{churchill2012marketing} explicam que:

\begin{citacao}
Embora a orientação para a produção seja muito criticada
por vários profissionais de marketing, há situações em
que ela é apropriada. Por exemplo, em mercados de alta
tecnologia com mudanças rápidas, muitas vezes não há
tempo suficiente realizar pesquisas de marketing a fim de
perguntar aos clientes o que eles querem.
\end{citacao}

ou

\begin{citacao}
Embora a orientação para a produção seja muito criticada
por vários profissionais de marketing, há situações em
que ela é apropriada. Por exemplo, em mercados de alta
tecnologia com mudanças rápidas, muitas vezes não há
tempo suficiente realizar pesquisas de marketing a fim de
perguntar aos clientes o que eles querem \parencite{churchill2012marketing}.
\end{citacao}

\subsection{Citação indireta}

Na visão de \textcite{osterwalder2011business} deve existir preocupação
com custos em quaisquer modelos de negócios, contudo as estruturas de baixo
custo são mais importantes para alguns tipos de negócios. Podemos dividir as
estruturas em duas: negócios direcionados pelos custos e negócios direcionados
pelo valor.

ou

A preocupação com custos é inerente a quaisquer modelos de negócios, contudo
as estruturas de baixo custo são mais importantes para alguns tipos de negócios.
Podemos dividir as estruturas em duas: negócios direcionados pelos custos e negócios
direcionados pelo valor \parencite{osterwalder2011business}.

\subsection{Citação de citação}

Para \textapud{simon1997administrative}[161]{mintzberg2010safari}  “o segredo da
resolução de problemas é que não existe segredos. Ela é realizada por meio de
complexas estruturas de elementos simples e conhecidos”.

\subsection{Citações de informações não publicadas}

No texto:

A empresa mantinha sete centros de revenda, prestando serviços exclusivos para duas
fábricas de produção de plástico biodegradável instaladas nas cidades de Diadema, São
Bernardo e Santo André\footfullcite{biolife2008}.

ou

Conforme relatório de descrição de atividade comercial da  \citeauthor{biolife2008}
publicado em \citedate{biolife2008}, a empresa mantinha sete centros de revenda,
prestando serviços exclusivos para duas fábricas de produção de plástico biodegradável
instaladas nas cidades de Diadema, São Bernardo e Santo André.

\subsection{Citação de fontes não paginadas}

De acordo com \textcite[cap. 1]{cnseg2017mercado}, durante os 800 anos que se passaram,
as duas atividades (seguro e bancária) permaneceram trocando conhecimentos entre si e
desenvolveram muito e até os nossos dias se aprimoraram mutuamente.

ou

Durante os 800 anos que se passaram, as duas atividades (seguro e bancária) permaneceram
trocando conhecimentos entre si e desenvolveram muito e até os nossos dias se aprimoraram
mutuamente \parencite[cap. 1]{cnseg2017mercado}\footnote{Fonte não paginada}.

\section{Mais alguns exemplos}

Referir-se ao arquivo \texttt{.tex} para ver como as citações são criadas.

\begin{itemize}
  \item \cite[ver:][30--42]{churchill2012marketing}
  \item \textcites[142-146]{churchill2012marketing}[21--23]{biolife2008}[45]{simon1997administrative}
  \item \cites[142-146]{churchill2012marketing}[21--23]{biolife2008}[45]{simon1997administrative}
  \item \cite{nouri2025}
  \item \cite{marginalrevolutionBordaCount}
  \item \apud{simon1997administrative}[161]{mintzberg2010safari}
\end{itemize}

    \chapter{Testando opções de glossário e índice}

Referência na forma \gls{latex}.

Referência na forma \gls{html}.

Abaixo, exemplo do \textit{template} canônico do pacote \texttt{glossaries}.

The \texttt{glossaries} package automatically generates a list of glossary entries. It's great for keeping track of your \gls{domain-knowledge} and \glspl{tla}. In this example we've put the glossary definitions in a separate \texttt{glossary.tex} file, which you can edit via the project menu.

In this example, several keywords\index{keywords} will be used 
which are important and deserve to appear in the Index\index{Index}.

Terms like generate\index{generate} and some\index{others} will also 
show up. Terms in the index can also be nested \index{Index!nested}

    \include{2-textuais/06-conclusão}

    % Elementos pós-textuais
    \postextual
    % Bibliografia (obrigatório)
    \printbibliography[title=REFER\^ENCIAS]
    % Glossário (opcional)
    \imprimirglossario
    % Apêndices (opcional)
    \begin{apendicesenv}
        \partapendices
        \chapter{Título (Opcional)}
Apêndice é tudo aquilo que foi \underline{produzido por quem escreveu o trabalho}, é necessário para entendimento de um argumento, fala ou dado, mas interromperia o fluxo de leitura caso fosse incluído no corpo do texto:  

\begin{itemize}
    \item Formulários 
    \item Questionários 
    \item Códigos de programação 
    \item Etc.
\end{itemize}
        \chapter{Kavookavala}
\lipsum
    \end{apendicesenv}
    % Anexos (opcional)
    \begin{anexosenv}
        \partanexos
        \chapter{Lavô tá Novo}
Anexo é tudo que \underline{não foi produzido por quem escreveu o trabalho}, é necessário para entendimento de um argumento, fala ou dado, mas interromperia o fluxo de leitura caso fosse incluído no corpo do texto: 

\begin{itemize}
    \item Análises 
    \item Relatórios 
    \item Gráficos e tabelas complexas e/ou longas 
    \item Conjuntos grandes de imagens 
\end{itemize}
        \chapter{Título (Opcional)}
\lipsum

    \end{anexosenv}
    % Índice remissivo (opcional)
    \imprimirindice
\end{document}